\chapter{Theory}\label{cha:intro}

\section{Machine learning}\label{sec:research:history}
Machine learning is a topic with many applications, especially useful for image classification. The concept of machine learning in general and neural networks in particular has been around for many decades, and has increased in popularity in recent years, owing to both breakthroughs in research and increase in hardware capabilities.
\subsection{Neural Networks}\label{sec:research:history}
\subsection{Convolutional Neural Networks}\label{sec:research:history}
In recent years, a kind of neural network has proven to be especially effective at image and video recognition, i.e. Convolutional Neural Networks (CNN). Instead of manually choosing which features to focus on when training a neural network, the CNN algorithm takes a whole image as input and finds these features automatically. This is done by convolving the image with trainable convolution kernels. Since the whole image can be used as input, a researcher does not have to worry about removing potentially useful data when manually pre-processing the image, which is a must when using other types of image classification algorithms.
Convolutional neural networks are inspired by the organization of the animal visual cortex, and are variations of multilayered neural networks.

\section{Problem outline}\label{sec:research:history}
