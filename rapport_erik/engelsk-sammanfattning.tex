The ratio between two different types of blood cells, i.e. heterophils and lymphocytes, is a useful measure to gauge the stress level of domestic chickens. This ratio is calculated by hand today, by manually counting blood cells in peripheral blood smear images. This is a very laborious and time consuming task, and prone to human error. This process should be possible to automate.\\\\
The aim of this thesis is to investigate automatic segmentation and classification of white blood cells in blood smear images taken from chickens in order to calculate the previously mentioned ratio. This is done through machine learning, by using Convolutional Neural Networks.\\\\
This thesis was produced at the AVIAN Behavioural Genomics and Physiology group at Linköping University, which provided blood smear image data from their chickens as well as expertise from their humans.\\\\
The results show that the process of calculating the ratio can be made semi-automatic or fully automatic, depending on the quality of the images and age of the individual chickens. 