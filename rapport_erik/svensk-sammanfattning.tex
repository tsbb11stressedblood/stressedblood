Förhållandet mellan två olika sorters vita blodceller, nämligen heterofiler och lymfocyter, är ett användbart mått på stressnivån hos kycklingar. Detta förhållande räknas ut för hand idag, genom att manuellt räkna ett antal blodceller i blodutstryk. Det är ett väldigt tidskrävande arbete, och även känsligt för misstag.\\\\
Detta examensarbete undersöker möjligheten att automatiskt segmentera och klassificera de vita blodcellerna i blodutstryk från kycklingar för att räkna ut detta förhållande. Detta görs genom maskininlärning, genom att använda så kallade Convolutional Neural Networks, faltande neuronnät.\\\\
Detta arbete genomförs i samarbete med AVIAN Behavioural Genomics and Physiology group på Linköpings Universitet, som tillhandahåller blodutstryk från sina kycklingar, och expertis från sina människor.\\\\
Resultatet visar att processen att räkna ut förhållandet mellan heterofiler och lymfocyter kan göras semi-automatiskt eller helautomatiskt, beroende på kvaliteten på bilderna och åldern på individerna.