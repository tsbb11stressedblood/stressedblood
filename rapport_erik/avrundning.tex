\chapter{Conclusion}\label{cha:conclusions}

\section{Conclusion}\label{sec:research:history}
This thesis project has proven to be more challenging than previously thought, for several reasons. The main reason is that the lymphocyte cells are simply too easy to be mistaken for broken red blood cells and platelets. The heterophils however are much easier to distinguish from other cells, which is reflected in the results. Although this semi-automatic method can be used to significantly lessen the burden on the researcher(s) tasked to count these cells.\\\\
A better method of counting the white blood cells of chickens is to use a flow cytometer\cite{Seliger201286}, although costly instruments prohibit this method from being widely used.

\section{Future Work}\label{sec:research:history}
There are many ways in which this method could be improved upon, the main ones are described in this section.

\subsection{Data gathering}\label{sec:research:history}
In order to generalize the algorithm further, much more data from chickens of different ages should be gathered. In this thesis, only chickens of the ages of 9 and 12 weeks were used. 